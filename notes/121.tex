%! Author = vladyslav
%! Date = 01.05.2022

% Preamble
\documentclass{article}

% Packages
\usepackage{amsmath}

\title{UniswapV2 - UniswapV3 arbitrage}
\author{Vladyslav Dalechyn}

\begin{document}

\maketitle


% Document
\section{Introduction}
    In order to run arbitrage, it's crucial to understand the underneath math in UniswapV2 and UniswapV3.\\

    UniswapV2 core is held in Pair, when UniswapV3 core is held in Pool.
    Pool is a tick-position-indexed state, and Pair is a regular state.

    We also know that the price of one asset can be defined as a price of the other asset.
    However, in case of UniswapV3, price is always defined as $token_1/token_0$.

\section{UniswapV2 Pair}
    Since all the data of a Pair is held in the fulfilled state which is available on contract,
    it's enough to read contract state to know the price, liquidity amounts, etc.

\section{UniswapV3 Pool}
    For the Pool situation is much different.\\
    We need to get the tick data, in which the liquidity data lies.

    When a swap is made in UniswapV3 Pool, the protocol looks if the current tick will be pushed
    up, or down, depending on the token ordering, which is a comparison of token contracts addresses and swap direction.

\end{document}